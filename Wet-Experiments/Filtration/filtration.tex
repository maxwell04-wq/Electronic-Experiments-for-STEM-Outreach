\chapter{Water Purification}

\section*{Learning Outcomes}
\small{
\begin{itemize}[leftmargin=*]\itemsep0em
\item What is raw water?
\item Why it is necessary to treat water?
\item How sand, gravel and charcoal will help to treat water?
\item What is sand filter?
\item What is Coagulation?
\item Why it is needed?
\item Most commonly used coagulant?

\end{itemize}
}
\section*{Introduction}
Raw water contains pathogens.\\[1mm]
{\large{\textbf{Coagulation}}} is the process of adding chemical (coagulant) such as Alum which produces positive charge with rapid mixing to neutralize the negative charge of particles so they come closer and form flocs. After rapid mixing we allow particles to come closer by giving time without rapid mixing.\\[1mm]
The selection of coagulant depends upon different factors such as temperature, pH, dosage and coagulant type. It has been observed that Low temperature and high pH slow down the coagulation process. Because low temperature slows down the reaction rate and It requires time to lessen the pH first by removal of particles by coagulation itself.Because maximum coagulation occurs at optimum pH ranges.\\[1mm]
We generally remove small particles by sedimentation but when particles are very small they take much time to settle down so for that purpose we use coagulation so that the particles may come closer together in very short time relatively and settle down. This is the primary process of any filtration process.\\[1mm]
The most commonly used coagulant is Alum(Aluminum sulphate) Al$_2$SO$_4$.\\[1mm]
Coagulation is done prior to filtration so that the efficiency of water treatment may increase.\\[1mm]
{\large{\textbf{Sand filter}}} is an environment friendly wastewater treatment method, which is relatively simple and inexpensive. Its principle involves percolating of water through sand bed. Raw water is the water which is untreated having minerals, electrolytes and pathogens as well. First of all to remove the large floating objects in water we need screening for this purpose we have gravel on the top which will not allow larger objects to pass through. Then we have layer of activated charcoal, it is not ordinary charcoal it means it has a lot of pores in it to capture the millions of smallest size particles. Sand allows water to percolate at faster rate as its permeability is very high and capture a lot of heavy metals like arsenic as well.\\[1mm]
It is for the treatment of water of low turbidity and having fewer amounts of other pollutants otherwise water will not be able to percolate through.

\section*{Demonstrations}

% Coagulation
\subsection*{Coagulation}

To make the demonstration, you will need:

\section*{Procedure}
\begin{enumerate}
    \item The water which we have treated through coagulation and flocculation now for further purification we will pass it through sand filter.In this way water will be more clear.
\end{enumerate}

\section*{Precautions}
\begin{itemize}[leftmargin=*]
\item 
\end{itemize}

\section*{Additional Notes}
\begin{itemize}[leftmargin=*]
\item 
 \end{itemize}

% Filtration
\subsection*{Filtration}
Filtration is the process of separating suspended solid matter from a liquid, by causing the latter to pass through the pores of some substance, called a filter. The liquid which has passed through the filter is called the filtrate.
It is a slow process. In case of Water Filtration depending factor is “Pours” present in the filter medium.

To make the demonstration, you will need:

\begin{table}[H]
    \centering
    \begin{tabular}{|c|l|}\hline
    1	&   Silt \\\hline
    2   &	Gravel \\\hline
    3   &  Fine Sand \\\hline
    4   &   Activated Charcoal\\\hline
    \end{tabular}
\end{table}

\section*{Procedure}
\begin{enumerate}
    \item 
\end{enumerate}

\section*{Precautions}
\begin{itemize}[leftmargin=*]
\item 
\end{itemize}

\section*{Additional Notes}
\begin{itemize}[leftmargin=*]
\item 
 \end{itemize}
